%-------------------------
% Resume in Latex
% Author : Matty
% Based on: https://github.com/jakegut/resume (which was itself based on https://github.com/sb2nov/resume)
% License : MIT
%------------------------

\documentclass[letterpaper,11pt]{article}

\usepackage{latexsym}
\usepackage[empty]{fullpage}
\usepackage{titlesec}
\usepackage{marvosym}
\usepackage[usenames,dvipsnames]{color}
\usepackage{verbatim}
\usepackage{enumitem}
\usepackage[hidelinks]{hyperref}
\usepackage{fancyhdr}
\usepackage[english]{babel}
\usepackage{tabularx}
\usepackage{xcolor}
\usepackage{fontawesome5}

\input{glyphtounicode}

% -------------------- FONT OPTIONS --------------------
% sans-serif
% \usepackage[sfdefault]{roboto}
% \usepackage[sfdefault]{noto-sans}
% serif
% \usepackage{charter}

\pagestyle{fancy}
\fancyhf{} % clear all header and footer fields
\fancyfoot{}
\renewcommand{\headrulewidth}{0pt}
\renewcommand{\footrulewidth}{0pt}

% Adjust margins
\addtolength{\oddsidemargin}{-0.5in}
\addtolength{\evensidemargin}{-0.5in}
\addtolength{\textwidth}{1in}
\addtolength{\topmargin}{-1in} % Default was -.5in
\addtolength{\textheight}{1.0in}

\urlstyle{same}

\raggedbottom
\raggedright
\setlength{\tabcolsep}{0in}

% Section formatting
\titleformat{\section}{
  \vspace{-5pt}\scshape\raggedright\large
}{}{0em}{}[\color{black}\titlerule \vspace{-5pt}]

% Subsection formatting
\titleformat{\subsection}{
  \vspace{-4pt}\scshape\raggedright\large
}{\hspace{-.15in}}{0em}{}[\color{black}\vspace{-8pt}]

% Ensure that generate pdf is machine readable/ATS parsable
\pdfgentounicode=1

% -------------------- CUSTOM COMMANDS --------------------
\newcommand{\resumeItem}[1]{
  \item\small{
    {#1 \vspace{-2pt}}
  }
}

\newcommand{\resumeSubheading}[4]{
  \vspace{-2pt}\item
    \begin{tabular*}{0.97\textwidth}[t]{l@{\extracolsep{\fill}}r}
      \textbf{#1} & #2 \\
      \textit{\small#3} & \textit{\small #4} \\
    \end{tabular*}\vspace{-7pt}
}

\newcommand{\resumeSubSubheading}[2]{
    \item
    \begin{tabular*}{0.97\textwidth}{l@{\extracolsep{\fill}}r}
      \textit{\small#1} & \textit{\small #2} \\
    \end{tabular*}\vspace{-7pt}
}

\newcommand{\resumeProjectHeading}[2]{
    \item
    \begin{tabular*}{0.97\textwidth}{l@{\extracolsep{\fill}}r}
      \small#1 & #2 \\
    \end{tabular*}\vspace{-7pt}
}

\newcommand{\resumeSubItem}[1]{\resumeItem{#1}\vspace{-4pt}}
\newcommand{\resumeSubHeadingListStart}{\begin{itemize}[leftmargin=0.15in, label={}]}
\newcommand{\resumeSubHeadingListEnd}{\end{itemize}}
\newcommand{\resumeItemListStart}{\begin{itemize}}
\newcommand{\resumeItemListEnd}{\end{itemize}\vspace{-5pt}}

\renewcommand\labelitemii{$\vcenter{\hbox{\tiny$\bullet$}}$}

\setlength{\footskip}{4.08003pt}

% -------------------- START OF DOCUMENT --------------------
\begin{document}

% -------------------- HEADING--------------------
\begin{flushright}
  % \vspace{-4pt}
  \color{gray}
  \item
  
\end{flushright}

\vspace{-5pt}

\begin{center}
    \textbf{\Huge \scshape Gustavo Angel Vargas Leal} \\ \vspace{3pt}
    \small 
    \faIcon{github}
    \href{https://github.com/GVL0001}{\underline{github.com/GVL0001}} $  $
    \faIcon{code}
    \href{https://gvl0001.github.io/portfolio/}
    {\underline{gvl0001.github.io/portfolio}} $  $
    \faIcon{linkedin}
    \href{https://linkedin.com/in/gustavo-angel-vargas-leal}{\underline{linkedin.com/in/gustavo-angel-vargas-leal}} $  $
    \mbox{\faIcon{envelope} \href{mailto:gustavovl0001@gmail.com}{\underline{gustavovl0001@gmail.com}}}
\end{center}

% -------------------- PERFIL PROFESIONAL --------------------
\section{Perfil Profesional}
    \resumeSubHeadingListStart
        QA Automation Engineer con experiencia en aseguramiento de la calidad. Especializado en pruebas manuales de nuevos desarrollos, diseño y ejecución de flujos automatizados (e2e y unitarias) para mantener la estabilidad del código. Stack actual: pruebas unitarias con Jest, automatización e2e con Playwright y Selenium; integración en CI con Actions. Conocimiento en testeo de APIs con Postman y automatización con Pytest. Experiencia previa en pruebas funcionales, UI, API REST y móviles; diseño de casos de prueba y seguimiento de bugs en Jira. Rápido aprendizaje de nuevas tecnologías y metodologías.
    \resumeSubHeadingListEnd

% -------------------- EXPERIENCE --------------------
\section{Experiencia}
  \resumeSubHeadingListStart

    \resumeSubheading
      {\textbf{Cynch} \href{https://cynch.me/}{\footnotesize [cynch.me]}}{Oct 2024 -- Presente}
      {QA Automation Engineer}{}
    \resumeItemListStart
      \resumeItem{Ejecución de pruebas manuales en nuevos desarrollos y creación de flujos automatizados para garantizar la estabilidad del código.}
      \resumeItem{Creación desde cero de suites e2e en áreas core; migración de pruebas e2e de Selenium a Playwright para mayor sostenibilidad.}
      \resumeItem{Desarrollo y mantenimiento de pruebas unitarias con Jest creadas desde cero.}
      \resumeItem{Integración de pruebas e2e en pipeline CI con GitHub Actions.}
    \resumeItemListEnd

  \resumeSubHeadingListEnd

% -------------------- PROJECTS --------------------
\section{Proyectos}
    \resumeSubHeadingListStart

    \resumeProjectHeading
{\textbf{Urban Routes - Web Testing Automated} $|$ \footnotesize\emph{Python, Pytest, Selenium WebDriver}}{Abril 2023}
\resumeItemListStart
\resumeItem{Desarrollé una suite completa de pruebas automatizadas utilizando Python, Pytest y Selenium para asegurar la calidad y el correcto funcionamiento de los principales flujos de la aplicación.}
\resumeItem{Implementé una arquitectura de Página de Objeto (POM) para mantener un código limpio, modular y facilitar el mantenimiento y la escalabilidad de las pruebas}
\resumeItem{Resolví desafíos técnicos aplicando las mejores prácticas de QA y asegurando la integridad y fiabilidad de las pruebas}
\resumeItemListEnd

    \begin{comment}
      \resumeProjectHeading
        {\textbf{Urban Grocers - API Testing Automated} $|$ \footnotesize\emph{Pytest, Requests, Python, Pycharm}}{Marzo 2024}
        \resumeItemListStart
            \resumeItem{Desarrollé un conjunto de pruebas automatizadas utilizando Python y el framework pytest para verificar la funcionalidad de creación de usuarios y creación de kits de usuario en una aplicación web}
            %\resumeItem{Utilicé la librería requests para enviar solicitudes HTTP a las API y validar las respuestas obtenidas}
            \resumeItem{Implementé casos de prueba exhaustivos, cubriendo escenarios positivos y negativos, asegurando el cumplimiento de los requisitos funcionales}
        \resumeItemListEnd
    \end{comment}

      \resumeProjectHeading
        {\textbf{Urban Lunch - Mobile Testing} $|$ \footnotesize\emph{Android Studio, Jira}}{Febrero 2024}
        \resumeItemListStart
            \resumeItem{Diseñé y ejecuté más 90 casos de pruebas móviles}
            \resumeItem{Reporté bugs y defectos encontrados en Jira}
            \resumeItem{Realicé pruebas en emuladores y dispositivos reales revisando los logs}
        \resumeItemListEnd

        \resumeProjectHeading
        {\textbf{Urban Grocers - API Testing} $|$ \footnotesize\emph{Postman, Jira, JSON, XML, ApiDoc, Swagger}}{Enero 2024}
        \resumeItemListStart
            \resumeItem{Manejé las pruebas de nuevas funcionalidades de API para gestión de kits, entregas y carrito}
            \resumeItem{Analicé requerimientos y diseñé más de 40 casos de prueba exhaustivos}
            \resumeItem{Ejecuté pruebas de API en Postman y reporté bugs encontrados en Jira}
          \resumeItemListEnd

        \resumeProjectHeading
        {\textbf{Urban Routes - Web Testing} $|$ \footnotesize\emph{JavaScript, Devtools, Jira, Charles Proxy, Figma}}{Diciembre 2023}
        \resumeItemListStart
            \resumeItem{Diseñé y ejecuté más de 100 pruebas de UI}
            \resumeItem{Identifiqué y reporté más de 50 bugs y defectos en Jira}
            \resumeItem{Realicé la interceptación de solicitudes API utilizando Charles Proxy}
          \resumeItemListEnd

    \begin{comment}
    \resumeProjectHeading
        {\textbf{Challenge Encriptador - Sitio web de encriptación de textos} $|$ \footnotesize\emph{HTML/CSS, JavaScript, VS Code}}{Marzo 2024}
        \resumeItemListStart
            \resumeItem{Desarrollé un sitio web para encriptar y desencriptar textos ingresados por el usuario}
            \resumeItem{Implementé diseños responsive entregados en Figma, asegurando una correcta visualización en desktop, tablets y mobile}
            \resumeItem{Configuré y desplegué el sitio web en una plataforma de hosting para su publicación y uso final}
        \resumeItemListEnd
    \end{comment}
          
    \resumeSubHeadingListEnd

    
% -------------------- EDUCATION --------------------
\section{Educación}
  \resumeSubHeadingListStart
  
    \resumeSubheading
      {Universidad Tamaulipeca}{Agosto 2020}
      {Técnico en programación}{}

    \resumeSubheading
        {TripleTen}{Mayo 2024}
        {QA Engineer}{}

    \vspace{-3pt}

  \resumeSubHeadingListEnd

% -------------------- FORMACION COMPLEMENTARIA --------------------

% -------------------- SKILLS --------------------
\section{Skills}
 \begin{itemize}[leftmargin=0.15in, label={}]
    \small{\item{
    
     \textbf{QA \& Testing}{: QA Manual, Pruebas unitarias (Jest), Pruebas e2e (Playwright, Selenium), Testeo de APIs (Postman), Automatización de APIs (Pytest + Postman)} \\
     
     \textbf{Lenguajes}{: TypeScript, JavaScript, Python, Java, HTML/CSS, SQL} \\
     
     \textbf{CI/CD \& Herramientas}{: GitHub Actions, Forgejo Actions, Git/GitHub, Cursor, VS Code, QASE, Charles Proxy, Postman, Jira, SQLite, PostgreSQL, Android Studio} \\
     
     \textbf{Frameworks}{: Jest, Playwright, Selenium WebDriver, Pytest} \\

     \textbf{Idiomas}{: Español, Inglés B1} \\
     
    }}
 \end{itemize}

\end{document}
